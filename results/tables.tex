\documentclass[a4paper,10pt]{article}
\usepackage{booktabs} 
\usepackage{dcolumn} 
\usepackage{graphicx}

\setlength{\textwidth}{6.5in}
\setlength{\oddsidemargin}{-.5in}
\setlength{\evensidemargin}{-.5in}

\begin{document}

%
% Figures
%

\begin{figure}[h!]
\centering
\includegraphics[width=0.7\textwidth]{figure1}
\caption{Decision tree to choose analysis approach for path models of constructs}
\end{figure}

\begin{figure}[h!]
\centering
\includegraphics[width=1.1\textwidth]{figure2}
\caption{Variance component model of optimal solution and summed scales}
\end{figure}

\begin{figure}[h!]
\centering
\includegraphics[width=1.1\textwidth]{figure3}
\caption{Variance component model of PLS analysis}
\end{figure}

\begin{figure}[h!]
\centering
\includegraphics[width=1\textwidth]{figure4}
\caption{Comparison of parameter estimates with PLS, summed scales, and  SEM}
\end{figure}

\begin{figure}[h!]
\centering
\includegraphics[width=1\textwidth]{figure5}
\caption{Estimated R2 over true R2}
\end{figure}

\begin{figure}[h!]
\centering
\includegraphics[width=1\textwidth]{figure6}
\caption{Cumulative probability of estimated p-values}
\end{figure}

%
% Tables
%

% Comparison of PLS with alternatives

\begin{table}[ht]
\begin{center}
\caption{Comparison of three methods to estimate path models}
\begin{tabular}{p{.25\textwidth}p{.25\textwidth}p{.25\textwidth}p{.25\textwidth}}
\toprule
&\multicolumn{1}{c}{PLS}&\multicolumn{1}{c}{Summed scales}&\multicolumn{1}{c}{SEM}\\
&&\multicolumn{1}{c}{and regression}\\
\midrule
Estimating construct values&Construct values are estimated as weighted linear combinations of indicators.&Construct values are estimated as unweighted linear combination of indicators.&Construct values are not estimated. Constructs are treated as latent variables. \\ \noalign{\smallskip}
Calculating path coefficients between the constructs&Separate OLS regression is run for each endogenous variable.&Separate OLS regression is run for each endogenous variable.& Model is expressed as a system of equations that is solved approximately by minimixing a fit function. \\ \noalign{\smallskip}
Estimating standard errors for the model parameters&Standard errors are estimated by resampling the entire model.&Standard errors are estimated based on known probability distribution.&Standard errors are estimated based on known probability distribution. \\ 
\bottomrule
\footnotesize{Typical cases}\\
\end{tabular}
\end{center}
\end{table}


% Experimental conditions

\begin{table}[ht]
\begin{center}
\caption{Experimental conditions}
\begin{tabular}{lp{.25\textwidth}p{.2\textwidth}p{.5\textwidth}}
\toprule
&\multicolumn{1}{c}{Condition}&\multicolumn{1}{c}{Values}&\multicolumn{1}{c}{Rationale} \\
\midrule
1&Number of constructs in the model&4, 8, 12&Since PLS takes the structural model into account when estimating the construct scores, model complexity can affect the results. \\ \noalign{\smallskip}
2&Expected number of outgoing paths&1, 2, 3 \\ \noalign{\smallskip}
3&Population path values&[-1,1],[0,1],[0]&PLS is often used to test models where all paths are positive. We wanted to test how it works if the pats are all zero or positive and negative with equal probability \\ \noalign{\smallskip}
\midrule
4&Sample size&50, 100, 250&Although large sample size is required for parameter consistency, PLS is often argued to work well with small sample sizes. \\ \noalign{\smallskip}
\midrule
5&Number of indicators per construct &3, 4, 6&Consistency at large criterion implies that the accuracy of PLS estimates depend on the number of indicators. \\ \noalign{\smallskip}
6&Mean factor loading&.4, .6, .8&Reliability affects attenuation and also indirectly bias. \\ \noalign{\smallskip}
7&Factor loading variation&0, .1, .2&PLS is assumed to be able to weight more reliable indicators more, so we needed to vary the indicator reliabilities. \\ \noalign{\smallskip}
8&Error correlations&0, .2, .4&PLS requires that there are no error correlations, although these are practically always present in the data. \\ \noalign{\smallskip}
9&Method variance&0, .15, .3&PLS has been shown to be affected by common method variance. This is always potentially an issue with survey studies. \\ \noalign{\smallskip}
\midrule
10&Omitted paths&0\%, 25\%, 50\%&Misspecified models are needed for testing if the method generates false positives. \\ \noalign{\smallskip}
11&Expected number of extra paths&0,1,2 \\
\bottomrule
\end{tabular}
\end{center}
\end{table}


\begin{table}[ht]
\begin{center}
\caption{Statistics on measurement quality}
\begin{tabular}{lrrrr}
 \toprule
 & Sum sc.& Comp. & Factor & PLS \\
\midrule
%See http://newsgroups.derkeiler.com/Archive/Comp/comp.text.tex/2008-11/msg00090.html
\csname @@input\endcsname table3_body.tex
\bottomrule
\end{tabular}
\end{center}
\end{table}

% True scores correlations

\begin{table}[ht]
\begin{center}
\caption{Reliability and bias of measurement}
\begin{tabular}{lrrrrrrrrrr}
\toprule
&\multicolumn{5}{c}{Reliability}&\multicolumn{5}{c}{Bias} \\
\cmidrule(l{.75em}){2-6}\cmidrule(l{.75em}){7-11}
\multicolumn{1}{c}{Analysis}&\multicolumn{1}{c}{5\%}&\multicolumn{1}{c}{50\%}&\multicolumn{1}{c}{95\%}&\multicolumn{1}{c}{H}&\multicolumn{1}{c}{L}&\multicolumn{1}{c}{5\%}&\multicolumn{1}{c}{50\%}&\multicolumn{1}{c}{95\%}&\multicolumn{1}{c}{H}&\multicolumn{1}{c}{L} \\
\midrule
\input{table4_body.tex}
\bottomrule
\multicolumn{11}{l}{\footnotesize{Distribution of statistics over all designs and replications}} \\
\multicolumn{11}{l}{\footnotesize{H/L = count of designs where the mean statistic for this analysis was higher/lower than others}} \\
\end{tabular}
\end{center}
\end{table}

%Measurement stability

\begin{table}[ht]
\begin{center}
\caption{Stability of measurement over data and models}
\begin{tabular}{lrrrrrrrrrr}
\toprule
&\multicolumn{5}{c}{Construct score}&\multicolumn{5}{c}{Construct scores} \\
&\multicolumn{5}{c}{SD over data}&\multicolumn{5}{c}{SD over models} \\
\cmidrule(l{.75em}){2-6}\cmidrule(l{.75em}){7-11}
\multicolumn{1}{c}{Analysis}&\multicolumn{1}{c}{5\%}&\multicolumn{1}{c}{50\%}&\multicolumn{1}{c}{95\%}&\multicolumn{1}{c}{H}&\multicolumn{1}{c}{L}&\multicolumn{1}{c}{5\%}&\multicolumn{1}{c}{50\%}&\multicolumn{1}{c}{95\%}&\multicolumn{1}{c}{H}&\multicolumn{1}{c}{L} \\
\midrule
\input{table5_body.tex}
\bottomrule
\multicolumn{11}{l}{\footnotesize{Distribution of statistics over all designs and replications}} \\
\multicolumn{11}{l}{\footnotesize{H/L = count of designs where the mean statistic for this analysis was higher/lower than others}} \\
\end{tabular}
\end{center}
\end{table}

% Correlations

\begin{table}[ht]
\begin{center}
\caption{Attenuation, bias, and error of correlations}
\begin{tabular}{lrrrrrrrrrrrrrrr}
\toprule
&\multicolumn{5}{c}{Attenuation coefficient}&\multicolumn{5}{c}{Bias}&\multicolumn{5}{c}{Error} \\
\cmidrule(l{.75em}){2-6}\cmidrule(l{.75em}){7-11}\cmidrule(l{.75em}){12-16}
\multicolumn{1}{c}{Analysis}&\multicolumn{1}{c}{5\%}&\multicolumn{1}{c}{50\%}&\multicolumn{1}{c}{95\%}&\multicolumn{1}{c}{H}&\multicolumn{1}{c}{L}&\multicolumn{1}{c}{5\%}&\multicolumn{1}{c}{50\%}&\multicolumn{1}{c}{95\%}&\multicolumn{1}{c}{H}&\multicolumn{1}{c}{L}&\multicolumn{1}{c}{5\%}&\multicolumn{1}{c}{50\%}&\multicolumn{1}{c}{95\%}&\multicolumn{1}{c}{H}&\multicolumn{1}{c}{L}  \\
\midrule
\input{table6_body.tex}
\bottomrule
\multicolumn{16}{l}{\footnotesize{Distribution of statistics over all designs and replications}} \\
\multicolumn{16}{l}{\footnotesize{H/L = count of designs where the mean statistic for this analysis was higher/lower than others}} \\
\end{tabular}
\end{center}
\end{table}

\begin{table}[ht]
\begin{center}
\caption{Average relative error and estimated standard error of regression}
\begin{tabular}{lrrrrrrrrrr}
\toprule
&\multicolumn{5}{c}{Average}&\multicolumn{5}{c}{Estimated} \\
&\multicolumn{5}{c}{relative error}&\multicolumn{5}{c}{standard error} \\
\cmidrule(l{.75em}){2-6}\cmidrule(l{.75em}){7-11}
\multicolumn{1}{c}{Analysis}&\multicolumn{1}{c}{5\%}&\multicolumn{1}{c}{50\%}&\multicolumn{1}{c}{95\%}&\multicolumn{1}{c}{H}&\multicolumn{1}{c}{L}&\multicolumn{1}{c}{5\%}&\multicolumn{1}{c}{50\%}&\multicolumn{1}{c}{95\%}&\multicolumn{1}{c}{H}&\multicolumn{1}{c}{L} \\
\midrule
\input{table7_body.tex}
\bottomrule
\multicolumn{11}{l}{\footnotesize{Distribution of statistics over all designs and replications}} \\
\multicolumn{11}{l}{\footnotesize{H/L = count of designs where the mean statistic for this analysis was higher/lower than others}} \\
\end{tabular}
\end{center}
\end{table}

\begin{table}[ht]
\begin{center}
\caption{Type I and Type II error rate at p$<$=.05}
\begin{tabular}{lrrrrrrrrrr}
\toprule
&\multicolumn{5}{c}{Type I rate}&\multicolumn{5}{c}{Type II rate} \\
\cmidrule(l{.75em}){2-6}\cmidrule(l{.75em}){7-11}
\multicolumn{1}{c}{Analysis}&\multicolumn{1}{c}{5\%}&\multicolumn{1}{c}{50\%}&\multicolumn{1}{c}{95\%}&\multicolumn{1}{c}{H}&\multicolumn{1}{c}{L}&\multicolumn{1}{c}{5\%}&\multicolumn{1}{c}{50\%}&\multicolumn{1}{c}{95\%}&\multicolumn{1}{c}{H}&\multicolumn{1}{c}{L} \\
\midrule
\input{table8_body.tex}
\bottomrule
\multicolumn{11}{l}{\footnotesize{Distribution of statistics over all designs and replications}} \\
\multicolumn{11}{l}{\footnotesize{H/L = count of designs where the mean statistic for this analysis was higher/lower than others}} \\
\end{tabular}
\end{center}
\end{table}

\begin{table}[ht]
\begin{center}
\caption{Type I and Type II error rate at p$<$=.01}
\begin{tabular}{lrrrrrrrrrr}
\toprule
&\multicolumn{5}{c}{Type I rate}&\multicolumn{5}{c}{Type II rate} \\
\cmidrule(l{.75em}){2-6}\cmidrule(l{.75em}){7-11}
\multicolumn{1}{c}{Analysis}&\multicolumn{1}{c}{5\%}&\multicolumn{1}{c}{50\%}&\multicolumn{1}{c}{95\%}&\multicolumn{1}{c}{H}&\multicolumn{1}{c}{L}&\multicolumn{1}{c}{5\%}&\multicolumn{1}{c}{50\%}&\multicolumn{1}{c}{95\%}&\multicolumn{1}{c}{H}&\multicolumn{1}{c}{L} \\
\midrule
\input{table9_body.tex}
\bottomrule
\multicolumn{11}{l}{\footnotesize{Distribution of statistics over all designs and replications}} \\
\multicolumn{11}{l}{\footnotesize{H/L = count of designs where the mean statistic for this analysis was higher/lower than others}} \\
\end{tabular}
\end{center}
\end{table}


\begin{table}[ht]
\begin{center}
\caption{Proportions of conditions where PLS had the best mean value of a statistic}
\begin{tabular}{lrrrrrrrrrr}
 \toprule
&\multicolumn{2}{c}{Constructs}&\multicolumn{2}{c}{Correlations}&\multicolumn{2}{c}{Regressions}&\multicolumn{2}{c}{Type I error}&\multicolumn{2}{c}{Type II error}\\
\cmidrule(l{.75em}){2-3}\cmidrule(l{.75em}){4-5}\cmidrule(l{.75em}){6-7}\cmidrule(l{.75em}){8-9}\cmidrule(l{.75em}){10-11}
&\multicolumn{1}{c}{Bias}&\multicolumn{1}{c}{SD over}&\multicolumn{1}{c}{Bias}&\multicolumn{1}{c}{Error}&\multicolumn{1}{c}{ARE}&\multicolumn{1}{c}{SE}&\multicolumn{1}{c}{p$<$=.05}&\multicolumn{1}{c}{p$<$=.01}&\multicolumn{1}{c}{p$<$=.05}&\multicolumn{1}{c}{p$<$=.01}\\
&&\multicolumn{1}{c}{data}\\
\midrule
%See http://newsgroups.derkeiler.com/Archive/Comp/comp.text.tex/2008-11/msg00090.html
\csname @@input\endcsname table10_body.tex
\bottomrule
\end{tabular}
\end{center}
\end{table}

\end{document}